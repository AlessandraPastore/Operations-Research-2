%************
%* packages *
%************

%Codifica
\usepackage[utf8]{inputenc}

%Lingua, sostituire italian con english nel caso in cui la tesi sia scritta in inglese
\usepackage[italian]{babel}

%Pacchetto per definire layout di pagina
\usepackage{fancyhdr}
\usepackage{sectsty}
\usepackage[left=3cm, right=3cm, bottom=3cm]{geometry}

%Spazia linee all'interno del documento
\usepackage{setspace}

%Listati di codice
\usepackage{verbatim}
\usepackage{listings}

%Didascalie immagini
\usepackage[hang,small,sf,font=scriptsize, labelfont=bf]{caption}
\usepackage{subcaption}

%Inclusione immagini
\usepackage{graphicx}

%Impostazioni note a piè pagina
\usepackage[stable]{footmisc}

%Citazioni e riferimenti a label
\usepackage{cite}
\usepackage[english]{varioref}

%Colori
\usepackage[usenames]{color}
\usepackage{xcolor}
\usepackage{colortbl}

%Crea link ipertestuali
\usepackage[hidelinks]{hyperref}

%Formattazione url
\usepackage{url}

%Inserimento formule
\usepackage{amsmath}
\usepackage{mathrsfs}

%Inserimento pseudocodice
\usepackage{algorithm}
\usepackage{algpseudocode}

%Citazione frasi
\usepackage{csquotes}

%Inserimento di Lorem ipsum nel testo
\usepackage{lipsum}

%\usepackage{lmodern}
\usepackage{amsfonts}

\usepackage{tikz}
\usepackage{tikzscale}

\usetikzlibrary{fadings}
\usetikzlibrary{patterns}
\usetikzlibrary{shadows.blur}
\usetikzlibrary{shapes}
\usetikzlibrary{fit}
\usetikzlibrary{positioning}
\usetikzlibrary{matrix}

