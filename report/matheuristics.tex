Matheuristics refers to the integration of Mathematical Programming and problem-specific heuristics to develop hybrid algorithms that leverage the strengths of both approaches. Many real-world optimization problems are computationally challenging and may not be efficiently solvable using exact mathematical programming approaches alone. This is where heuristics come into play. Heuristics are approximate methods that trade optimality for computational efficiency. They aim to quickly find good solutions by leveraging problem-specific knowledge, rules of thumb, or iterative improvement techniques.
The combination of mathematical programming and heuristics in matheuristics allows for a powerful approach to tackle complex optimization problems, providing a good balance between solution quality and computational efficiency. It is particularly useful for problems, such as TSP, where finding exact optimal solutions is infeasible or time-consuming, but good-quality solutions are still desired within a reasonable amount of time.
In the following section we present two of such techniques, that are hard fixing and local branching.

\section{Hard Fixing}
