\section{History of TSP}
The Traveling Salesman Problem (TSP) is one of the most well-known and challenging problems in computer science, operations research, and mathematics. The problem is often referred to as the "mother of all optimization problems."

The TSP is a problem in which a salesman must visit a certain number of cities, each only once, and return to the starting city in the most efficient way possible. The goal is to find the shortest possible route that visits all cities exactly once and returns to the starting city.

The TSP was first formulated in the 1800s by the Irish mathematician Sir William Hamilton and the British mathematician Thomas Kirkman. 

In the 1930s, the TSP was formulated as a mathematical problem by the Hungarian mathematician Péter Kórmán. He proved that the problem is NP-hard, which means that it is unlikely that there is a polynomial-time algorithm to solve it.

The first practical algorithm for the TSP was developed by George Dantzig, Ray Fulkerson, and Selmer Johnson in the late 1940s. They used linear programming to solve the problem for small instances.

In the 1950s and 1960s, several heuristics were developed for the TSP, including the nearest neighbor algorithm, the insertion algorithm, and the 2-opt algorithm. These heuristics provided fast and approximate solutions for the TSP.

In the 1970s, the branch and bound algorithm was developed, which provided an exact solution to the TSP for small instances. However, this algorithm became impractical for larger instances.

In the 1980s and 1990s, several metaheuristics were developed for the TSP, including simulated annealing, tabu search, genetic algorithms, and ant colony optimization. These algorithms provided good solutions for the TSP for larger instances.

Today, the TSP remains an active area of research, and many new algorithms and techniques are being developed to solve the problem. The TSP has applications in many fields, including logistics, transportation, and network optimization.

\section{Problem formulation}

Given a directed graph G=(V,A) the Traveling Salesman Problem consists in finding a Hamiltonian circuit of a minimum cost. This problem arises in various real-world applications, whenever there is a need to find an optimal or near-optimal route for visiting a set of locations.\\
The problem can also be defined on an undirected graph; this happens when the cost associated with an arc does not depend on its orientation.\\ 
In our work we put our attention on the \textit{Symmetric Traveling Salesman Problem}. In this scenario there is an undirected weighted complete graph G=(V,E) where V=\{$v_1,\dotsc,v_n$\} is the set of nodes, and E is the set of edges. Every edge has an associated cost, let c: E $\rightarrow \mathbb{R}^+$ be a function that assigns to every edge a proper cost.\\
The goal of TSP is to find the shortest tour that connects all the nodes. The tour can be formulated as a sequence of edges, or nodes. In order to solve this problem, there is the need to formulate it as a mathematical problem.
\\Using the following integer variables:



\begin{equation*}
    x_e=
        \begin{cases}
            \text{1 if edge e $\epsilon$ E is selected}\\
        0 \text{ otherwise}\\
          \end{cases}
\end{equation*}

we obtain the following model:

\begin{equation}
    \text{min}\sum_{e \epsilon E}^{}c_ex_e
\end{equation}

\begin{equation}
    \sum_{e \epsilon \delta (v)}^{}x_e=2 \text{ $\forall$\textit{v} $\epsilon$V } 
\end{equation}

\begin{equation} \label{subeq}
    \sum_{e\epsilon E(S)}^{}x_e \leq |S|-1 \text{ $\forall$S $\subset$ V : |\textit{S}| $\geq$ 3}
\end{equation}

\begin{equation}
    x_e \epsilon \{0,1\} \text{ $\forall$\textit{e} $\epsilon$ \textit{E}}
\end{equation}


The constraint \ref{subeq} is defined as Subtour elimination constraint (SEC). Without this constraint, the solution to the problem does not consist of a single circuit, but of several sub-circuits. Thus it imposes that the final solution returned by the problem forms a single Hamiltonian circuit.