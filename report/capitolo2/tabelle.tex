Nella seguente tabella vengono mostrati alcuni esempi di separazione delle righe. La separazione delle colonne avviene  all'interno delle parentesi grafe dopo il comando \textit{tabular}.
\begin{center}
\begin{tabular}{ ||c|c|c|| } 
 \hline
 cella1 & cella2 & cella3 \\
 \hline\hline 
 cella4 & cella5 & cella6 \\ 
 \hline
 cella7 & cella8 & cella9 \\ 
 cella10 & cella11 & cella12 \\ 
 \hline
\end{tabular}
\end{center}

È possibile, inoltre, fissare la dimensione delle colonne.

\begin{center}
\begin{tabular}{ | m{3cm} | m{5cm} | m{1.5cm} | } 
 \hline
 cella1 & cella2 & cella3 \\
 \hline\hline 
 cella4 & cella5 & cella6 cella6 cella6 cella6 \\ 
 \hline
 cella7 & cella8 & cella9 \\ 
 cella10 & cella11 & cella12 \\ 
 \hline
\end{tabular}
\end{center}